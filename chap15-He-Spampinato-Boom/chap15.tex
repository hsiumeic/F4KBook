This chapter describes the algorithms and tools we have developed
to collect ground truth data for training and evaluating
each of the system components.
 
\section{Detection and Tracking}
%Three tools have been devised to generate ground truth for detection and tracking: the first one is standard stand-alone tool that exploits computer vision techniques to support users for fast generation of ground truth, the second one, instead, is a collaborative tool that integrates the contributions of motivated users to generate high-quality ground truth, and, the last one is an online game that resorts to the crowd to generate reliable ground truth for object detection and tracking.
Each of the aforementioned tool is explained in the following.


Three tools have been devised to generate ground truth for detection and tracking: the first one is standard stand-alone tool that exploits computer vision techniques to support users for fast generation of ground truth, the second one, instead, is a collaborative tool that integrates the contributions of motivated users to generate high-quality ground truth, and, the last one is an online game that resorts to the crowd to generate reliable ground truth for object detection and tracking.
Each of the aforementioned tool is explained in the following.

\section{Species recognition}

A discussion of the problem, the general idea of converting a
problem that needs specialists' knowledge to a problem that can
solved by laymen. 

\subsection{A cluster based approach to fish recognition}
input from the ICPR paper

\subsection{Expert, non-expert, and automatic method}
%
Crowd-sourcing has shown to provide effective solutions to many
labeling tasks.  However, tasks in specialist domains are difficult to map to
Human Intelligence Tasks (or HITs) that can be solved adequately by "the
crowd". The question addressed in this paper is whether these specialist tasks
can be cast in such a way, that accurate results can still be obtained through
crowd-sourcing.
%
We study a case where the goal is to identify fish species in images extracted
from videos taken by underwater cameras, a task that typically requires
profound domain knowledge in marine biology and hence would be difficult, if
not impossible, for the crowd. 
%
We show that by carefully converting the recognition task to a visual
similarity comparison task, the crowd achieves agreement with the experts
comparable to the agreement achieved among experts.  Further, non-expert users
can learn and improve their performance during the labeling process, e.g., from
the system feedback.

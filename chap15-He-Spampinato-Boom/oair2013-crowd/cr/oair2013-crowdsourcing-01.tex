\section{Introduction}
\label{sec:intro}
%\todo{Terminology: user = non-expert annotator; game or labeling
%interface; labeling is an action that provides labels, each image has
%multiple labels (non-zero vote candidate categories), 
%change candidate category to candidate label? a judgement is a 0/1
%decision whether a candidate label/category is relevant.} \todo{Check
%terminology: annotation, labeling, label, judgements}
Creating ground truth data for video-based retrieval and computer vision research 
is often a time consuming task done by humans using
dedicated tools such as those presented in~\cite{Spam12:vigta}.
%
Recently, crowd-sourcing as a collaborative problem solving strategy has
received much attention.
In particular, within the computer vision communities, where \emph{large scale}
ground truth data are needed, the wisdom of the crowd was shown to
provide effective solutions in a wide range of problems~\cite{Russell08:Label, 
Yuen09:Label, ahnl:04, ahnl06:peekaboom, Chen:2011:LFA}. %anhl06:impr
%
Typically, the annotation/labeling tasks the crowd is asked to 
perform are relatively easy, that is, little or no expert knowledge is required. 

Instead, this paper studies an image labeling task that requires
highly specialized domain knowledge. %The ground truth obtained in this manner 
The ground truth obtained
serves as training material for machine learning approaches
that aim to classify fish species on video footage of Taiwanese coral
reefs.
%%
%This is a difficult task, both because the footage is often of
%relative low quality (bad lighting conditions, murky water) and
%because many fish species are visually very similar.  More
%importantly, 
Correctly identifying fish species %by their scientific
%name 
on this footage requires expertise from marine biologists, 
which is highly localized: biologists specialized on the
Australian reefs perform not as good as those specialized on the
%fishes that live on the Taiwanese reefs). 
Taiwanese coral reef fish species.
%
%On the other hand, 
Further, since experts are a scarce and expensive resource, 
%it is not likely to rely on them to obtain
it is unlikely that they would provide 
the amount of image labels needed for the purpose
of training and evaluating the fish classification models. 
The question is then, can we create a ground truth set of sufficient \emph{quantity} with sufficient \emph{quality}
by taking advantage of the collaborative problem solving ability of the crowd, %strategy, %ability of the crowd, 
while solving the problem that the crowd generally lacks the domain knowledge required by the task?

A smart way of presenting a problem or decomposing a complicated problem 
into simpler sub-problems may greatly reduce its difficulty and makes an infeasible task feasible. 
Typical examples include Foldit~\cite{cooper2010:pred} that
uses a puzzle solving game for protein structure prediction.
%Their results show that it is possible to use non-experts to solve complicated
%scientific problems.
Another example is Galaxy zoo~\cite{website:Galaxyzoo} 
%\footnote{\url{http://www.galaxyzoo.org/}}
that uses ``citizens' wisdom''
to contribute to morphological classification of galaxies.
%%
%Similar to these problems, our image labeling task requires highly specialized knowledge in marine biology. Since experts are a scarce and expensive resource and are not likely to provide labels in large quantity, 
%Because experts are a scarce resource, 
%we use their expertise to transform the difficult fish labeling task into a game based on a
%visual similarity comparison task that can be performed by large numbers of non-experts. 
%Because experts are a scarce resource, 
%
%With respect to 
For our labeling problem,  we use the expertise of marine biologists to
transform the fish identification task into a game based on a
visual similarity comparison task that can be performed by a large
number of non-experts. 
%
%In the game, players are shown a single
%\emph{query image} along with multiple labeled images of candidate
%species, referred to as \emph{candidate labels}, and are asked to
%assign the query image to the label that depicts the same species as
%the fish in the image. 
We then conduct a user study and seek the 
answers to the following questions:
(i) Can non-expert players of this game achieve acceptable
performance evaluated with the labels provided by the experts?
and (ii) Can players learn and improve their performance during the game?
%Can the performance of non-experts be improved through
%learning during the game?
%we first question whether non-expert players of this
%game can achieve acceptable performance when compared with the
%experts' performance in the original recognition task. Second, we
%analyze to which extent this performance can be learned during the
%game.
%
We find that after the task conversion, non-expert players achieve an
agreement with the experts comparable to that achieved among the
experts themselves. Further, players improve their performance while
playing the game: they are able to recognize a fish better not only
when they see the same fish again, but also when
they see a different fish from the same species. 

Our contributions are two fold. First, we propose a task conversion
approach to solve an image recognition problem that requires highly
specialized domain knowledge with non-expert users. Second, our study
on the learning behavior of the non-expert users provides insights
into the ability and limits of the crowd.  

%=================== OLD ===============================
\if 0
Many approaches in information retrieval (IR) and computer vision (CV)
rely on relevance judgements or correctly labeled images, both to
train and to evaluate the algorithms developed. 
%
Creating ground truth data %for video-based retrieval and computer
vision research is often a time consuming task done by humans using
dedicated  tools such as those presented in~\cite{Spam12:vigta}.
%
%\cite{Giro2012:multi, Moeh2012:effe}.  Examples of crowdsourcing such
%ground truth collection in order to obtain larger scale datasets
%include  \cite{Russell08:Label} for image and video labeling or
%\cite{Hoss12:aggr} for INEX document/topic relevance assessments.
%
Crowd-sourcing as a colleborative problem solving strategy has
received much attention recently.
%
In particular, within the IR and CV communities, where large scale
ground truth data are needed, the wisdom of the crowd was shown to
provide effective solutions in a wide range of problems, ranging from
image/video annotation~\cite{Russell08:Label, Yuen09:Label, ahnl:04,
anhl06:impr, ahnl06:peekaboom, Chen:2011:LFA}, to text
annotation~\cite{AMBATI10.244, Finin:2010:ANE} and search result
assessments~\cite{eickhoff12:qual, Hoss12:aggr}. %kazai:overview11
%
%
%Creating such ground truth data sets typically requires a large
%amount of manual effort.  Crowd-sourcing is a commonly applied
%strategy. 
%
Typically, the task the crowd is asked to perform is relatively easy,
and the focus is on the incentives needed to attract a sufficient
\emph{quantity}~\cite{ahnl:04, Russell08:Label} of users who together
create a dataset of sufficient \emph{quality}~\cite{Kazai09onthe,
Kazai11crowd, quinn11:survey}. 

%Instead, this paper studies a domain in which an image labeling task
Instead, this paper studies an image labeling task that requires
highly specific domain knowledge. The ground truth obtained in this
manner serves as training material for machine learning approaches
that aim to classify fish species on video footage of Taiwanese coral
reefs.
%
This is a difficult task, both because the footage is often of
relative low quality (bad lighting conditions, murky water) and
because many fish species are visually very similar.  More 
importantly, 
Correctly identifying fish species by their scientific
name on this footage requires expertise (i.e. from marine biologists)
which is highly localized (i.e. biologists specialized on the
Australian reefs perform not as good as those specialized on the
fishes that live on the Taiwanese reefs). 

\todo{Some reviewers ask why we don't get high quality images: this is
not a question, it doesn't make sense to train classifiers on high quality images
and apply it on low quality images.}

Because experts are a scarce resource, we use their expertise to
transform the difficult fish labeling task into a game based on a
visual similarity comparison task that can be performed by large
numbers of non-experts. In the game, players are shown a single
\emph{query image} along with multiple labeled images of candidate
species, referred to as \emph{candidate labels}, and are asked to
assign the query image to the label that depicts the same species as
the fish in the image. 

%In this paper, we first question whether non-expert players of this
%game can achieve acceptable performance when compared with the
%experts' performance in the original recognition task. Second, we
%analyze to which extent this performance can be learned during the
%game.
%
We ask two research questions:
\begin{description}
\item[RQ1.] Can non-expert players of this game achieve acceptable
performance evaluated with the labels provided by the experts?
\item[RQ2.] Can the performance of non-experts be increased through
learning during the game?
\end{description}
%with the experts' performance in the original recognition task

To study these research questions, we ask three Taiwanese marine
biologists to perform the recognition task and analyze the results.
%(Section~\ref{sec:expt_label}).  
%
We then transform the task into an image matching game that is played
by non-experts in two modes. First, we study players' performance in
an ideal setting where the correct answer is always present. 
%
We compare this to a more realistic setting, where in some cases the
correct species are not shown, but other, visually similar species
are.   
% (details of the game and both settings are given in
% Section~\ref{sec:nonexpt_label}). 
%(Section~\ref{sec:nonexpt_label}). 
%
We evaluate the game results in terms of (a) agreement between experts
and non-experts; (b) the quality of the non-expert labels measured by
NDCG; and (c) the learning behavior of the players in terms of
memorization and generalization. %(Section~\ref{sec:eval}).
%
%The result of the evaluation is discussed in Section~\ref{sec:res},
%while Section~\ref{sec:con} concludes the paper with a discussion of
%the implications, limitations and future work of our study. 
%
We find that after the task conversion, non-expert users achieve an
agreement with the experts comparable to that achieved among the
experts themselves. Further, players improve their performance while
playing the game, not only are they able to better recognize a fish
when they see the same fish (i.e., same image) again, but also when
they see a different fish (i.e., different image) of the same species. 

Our contributions are two fold. First, we propose a task conversion
approach to solve an image recognition problem that requires highly
specialized domain knowledge with non-expert users. Second, our study
on the learning behavior of the non-expert users provides insights
into the ability and limits of the crowd.  

The rest of the paper is organized as follows. 
We discuss related work in Section~\ref{sec:rel}. 
%on crowd sourcing for collecting ground truth data.
In Section~\ref{sec:expt_label} we describe our experiment with
experts for the recognition task. We present the details of the game
and our experiments with non-expert players
in~\ref{sec:nonexpt_label}. followed by our evaluation setup in
Section~\ref{sec:eval}. The results of this evaluation are presented
and discussed in Section~\ref{sec:res}. Section~\ref{sec:con}
concludes the paper with a discussion of the implications,
limitations, and future work of our study. 
\fi
%We then evaluate the results in terms of (a) agreement between experts and non-experts, (b) the quality of the non-
%expert ranking measured by NDCG, and (c) the learning effects measuring memorisation and generalisation 
%(Section \ref{sec:eval}).

%\todo{Crowdsourcing is a commonly applied strategy. 
%Many studies in crowd sourcing focuses on incentives. We focus on a different problem: knowledge.}
%
%\todo{We conduct a case study, ... introduce the task, why it is difficult.}
%
%
%\todo{What do we do?}
%\begin{itemize}
% \item convert a expert task to a non-expert task 
% \item provide feedback to non-experts and hopefully they can learn during the labeilng process. 
% \end{itemize}
%
%We study an image labeling case where the primary goal is to identify fish species
%from images extracted from videos recorded by underwater videos. 
%With respect to the two types of annotators, i.e., experts and non-experts, we consider
%two separate tasks, namely \textit{recognition} and \textit{similarity comparison}.
%
%The experts are asked to perform the recognition task. 
%That is, given an image, the expert is asked to provide the scientific name of the fish in the image. 
%The non-experts, on the other hand, are asked to perform the similarity comparison task. 
%They are shown examples of some candidate species, referred to as \emph{categories},  
%and assign the image to one of the categories that they believe has the same species as the fish in the image. 
%This way,  instead of actively come up with the species of the fish, the non-experts passively receive
%the possible candidate species, and make decisions based purely on comparisons on visual similarities.   
%This is a much easier task than the recognition task, feasible for people without any domain knowledge. 
%
%The question is then: \emph{are the labels obtained in this way of sufficient quality?}
%In order to answer this question, we conduct the following experiments. 

%\todo{What do we want to find out?}
%\begin{itemize}
%\item Can non-experts achieve a accetable performance in an expert task (given a sepicific setup)?
%\item Can non-experts learn and improve their performance?
%\end{itemize}
%
%Make aggregation part of the methods.

Chap19.tex
Please provide your chapter content here. 

Outline: 

What has been achieved ?

 - video capture 100K hours
 - video analysed: fish detect, track, recognised
 - 23 species, c. 85% accuaracy over species, 95% over instances
 - interactive query system, with on-demand computation
 - HPC for managing the capture, analysis and query
 - a generic workflow that is easily modifiable, fault tolerance
 - large database: 1+B detections of 100M fish 
 
What worked well:
 
- many good detections even with hard to analyse video
- recognition even with very unbalanced classes
- new methods for more efficient or fun ground-truthing
- lots of publications 

What lies in the future - 

- A brif summary of achievements of the project as a whole:
- What the project has done that would otherwise not being possible, 
  e.g data gathered but not being looked at/analysed
- This project is only a starting point for the marine biologists to 
  start looking and analysing mass recording videos in the wild 
- How such an approach may be repeated in other similar domains/problems
- To work with larger, faster, higher resoltuion videos
- Is real-time video analysis and user query answering a valid challenge?
- Is a generic system useful for marine science?
- Re-engineer the video analysis programs 
